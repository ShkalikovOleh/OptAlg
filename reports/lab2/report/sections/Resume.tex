\begin{center}
    \section*{Висновки}
\end{center}

У даній лабораторній роботі були реалізовані
$8$ методів оптимізації: метод Нелдера-Міда,
Гука-Дживса, Ньютона, SR1, Broyden, DFP, BFGS, генетичний алгоритм.
Усі реалізації було протестовано на $3$ тестових функціях,
які застосовуються для перевірки подібних методів,
а саме: Розенброка, Хіммельблау, Еклі.

Для метода Нелдера-Міда було побудовано графіки
історії збіжності до оптимумів(рис. 3), таблицю(таб. 2) результатів
запусків на цільових функціях та графіки значень функції(рис. 9) у кожнії точці
симплекса від номера ітерації алгоритму.
На обраних функціях цей метод показав значно кращий
рещультат, ніж інші методи оптимізації(у тому числі ті, які
використовують похідні та градієнти функції).

Для методів Хука-Дживса, Ньютона та квазіньютонівських було побудовано графіки
збіжності(рис. 4-8), порівняльні таблиці(таб. 3-6) з кількістю кроків,
необхідних для виконання зазначенної умови зупинки, та
графіки(рис. 13-15) значень функції на кожному кроці(у логарифмічній шкалі).
На основі цієї інформації можна зробити висновок, що
найкращим з точки зору збіжності(та швидкості збіжності) є
квазіньютонівський метод BFGS.

Для імунного генетичного алгоритму
було побудовано графіки історії антитіл (рис. 9-11) та
графіки значень функції(рис. 16) за номером популяції.

Також, хочемо зазначити, що у практичних задачах
можна комбінувати ті методи, що ми дослідили, для забезпечення
найбільш якісних та "швидких" результатів. Наприклад, дуже
ефективною є комбінація методу Гука-Дживса та генетичного алгоритму.
