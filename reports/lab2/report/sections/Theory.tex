\begin{center}
    \section*{Теоретична частина}
\end{center}
\SetKwInOut{Parametr}{}

Задача безумовної оптимізації скалярної функції $f(x)$
полягає у тому, щоб знайти такий $x_{min} \in D$
($ D \subseteq \mathbb{R}^n $ - область визначення $f$), що:

\begin{equation} \label{eq:optimization_task}
    x_{min} = \argmin\limits_{x \in D} f(x)
\end{equation}

Не для усіх функцій можливо просто та швидко
аналітично знайти екстремуми, тому у практичних
задачах дуже часто використовуються числені методи
оптимізації. У даній роботі ми розглянемо

\subsection*{Методи спуску}

Для оптимізації скалярної функцій можна
використовувати методи спуску, що на кожному кроці
будуть наближати нас до якогось локального мінімуму.
Їх можна описати наступною процедурої зміни
аргумента функції($\alpha_k > 0$, $k \in \mathbb{N}$):

\begin{equation} \label{eq:descent}
    x_{k+1} = x_{k} - \alpha_k p_{k}
\end{equation}
За умови, що виконується нерівність:
\begin{equation} \label{eq:descent_require}
    f(x_{k+1}) < f(x_k)
\end{equation}

Ітерації спуску продовжуются доки не виконана умова зупинки.
Серед умов зупинки можна виділити наступні($\varepsilon > 0,\;
n \in \mathbb{N}$ - параметри):

\begin{enumerate}
    \item Значення градієнту: $|\nabla f_k| < \varepsilon$
    \item Різниця аргументів: $|x_k - x_{k-1}| < \varepsilon$
    \item Різниця значення функції: $|f(x_k) - f(x_{k-1})| < \varepsilon$
    \item Кількість ітерацій більша або рівна за $n$
\end{enumerate}
Де $ \nabla f_{k} = \nabla f(x_{k})$(надалі будемо
використовувати це позначення).

\subsubsection*{Метод Ньютона}

\subsection*{Квазіньютоновські методи}

\subsection*{Метод Гука-Дживса}

\subsection*{Метод Нелдера-Міда}

\subsection*{Генетичний алгоритм}

\begin{algorithm}[H] \label{alg:newton}
    \SetAlgoLined
    \KwIn{$x_0$}
    \KwResult{$x_{min}$}
    $x_1 \leftarrow x_0$\;
    \While{не виконана умова зупинки}
    {
        $\alpha_k \leftarrow \argmin\limits_{\alpha > 0} f(x_{k} - \alpha H^{-1}_k \nabla f_{k})$\;
        $x_{k+1} \leftarrow x_{k} - \alpha_k H^{-1}_k \nabla f_{k}$\;
    }
    \KwOut{$x_k$}
 \caption{Метод Ньютона}
\end{algorithm}